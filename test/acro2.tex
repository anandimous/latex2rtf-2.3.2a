\documentclass{article}
\usepackage{hyperref}
\usepackage[printonlyused,withpage]{acronym}
\begin{document}
\section{Intro}
In the early nineties, \acs{GSM} was deployed in many European
countries. \ac{GSM} offered for the first time international
roaming for mobile subscribers. The \acs{GSM}'s use of \ac{TDMA} as
its communication standard was debated at length. And every now
and then there are big discussion whether \ac{CDMA} should have
been chosen over \ac{TDMA}.

\section{Furthermore}
\acresetall
The reader could have forgotten all the nice acronyms, so we repeat the
meaning again.

If you want to know more about \acf{GSM}, \acf{TDMA}, \acf{CDMA}
and other acronyms, just read a book about mobile communication. Just
to mention it: There is another \ac{UA}, just for testing purposes!

\begin{figure}[h]
Figure
\caption{A float also admits references like \ac{GSM} or \acf{CDMA}.}
\end{figure}

\subsection{Some chemistry and physics}
\label{Chem}
\ac{NAD+} is a major electron acceptor in the oxidation
of fuel molecules. The reactive part of \ac{NAD+} is its nictinamide
ring, a pyridine derivate.

One mol consists of \acs{NA} atoms or molecules. There is a relation
between the constant of Boltzmann and the \acl{NA}:
\begin{equation} 
k = R/\acs{NA}
\end{equation}

\acl{lox}/\acl{lh2} (\acsu{lox}/\acsu{lh2})

\section{Acronyms}

\begin{acronym}[TDMA]
\acro{CDMA}{Code Division Multiple Access}
\acro{GSM}{Global System for Mobile communication}
\acro{NA}[\ensuremath{N_{\mathrm A}}]{Number of Avogadro\acroextra{ (see \S\ref{Chem})}}
\acro{NAD+}[NAD\textsuperscript{+}]{Nicotinamide Adenine Dinucleotide} 
\acro{NUA}{Not Used Acronym} 
\acro{TDMA}{Time Division Multiple Access} 
\acro{UA}{Used Acronym}
\acro{lox}[\ensuremath{LOX}]{Liquid Oxygen}
\acro{lh2}[\ensuremath{LH_2}]{Liquid Hydrogen}
\end{acronym}
\end{document}